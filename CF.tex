\documentclass[a4paper]{article}

\usepackage[francais]{babel}
\usepackage[applemac]{inputenc}
\usepackage{fontenc}
\usepackage{amsthm}
\usepackage{amsfonts}
\usepackage{graphicx}
\usepackage{array}
\usepackage{amsmath}
\usepackage{mathenv}
\usepackage{a4wide}
%\usepackage{fullpage}

\newtheorem{prop}{Propri�t�}
\newtheorem{lemme}{Lemme}
\newtheorem{defi}{D�finition}
\newtheorem{cor}{Corollaire}
\newtheorem*{rmq}{Remarque}
\newtheorem*{ex}{Exemple}

\newcommand{\lis}[2]{#1_1,\;\dots,\;#1_{#2}}

\title{Calcul formel}
\author{Christophe Vuillot et Yannick Zakowski}
\begin{document}
\maketitle
 
 Consid�rons la courbe elliptique $\mathcal{E}$ d'�quation $y^2=x^3+ax+1.$
 
 Soit $P=(x_P,\; y_P)$ et $Q =(x_Q, \; y_Q)$ deux points de $\mathcal{E}.$\\
 
 La droite $(PQ)$ a pour �quation : 
 \[y=\lambda x+ y_P - \lambda x_P \text{ avec }\lambda = \frac{y_Q-y_P}{x_Q-x_P} \]
 
 Cherchons les coordonn�es du troisi�me point d'intersection de cette droite avec $\mathcal{E} : $\\
\begin{equation*}
\left\{
\begin{aligned}
	&y=\lambda x+ y_P - \lambda x_P \\
	&y^2=x^3+ax+1
\end{aligned}
\right.
\end{equation*}

Son abscisse v�rifie donc :
\begin{equation*}
\begin{aligned}
	&(\lambda x+ y_P - \lambda x_P)^2=x^3+ax+1\\
	&x^3 - \lambda^2x^2+(a+2\lambda^2x_P-2\lambda y_P)x+1=0\\
	x_P\text{ �tant solution }&(x-x_P)(x^2+(x_P-\lambda^2)x+(a+2\lambda^2x_P-2\lambda y_P+x_P(x_P-\lambda^2))=0\\
	x_Q\text{ �tant solution }&(x-x_P)(x-x_Q)(x+x_P+x_Q-\lambda^2)=0
\end{aligned}
\end{equation*}

$
\text{D'o� }\left\{
\begin{aligned}
	&x_{P+Q}=\lambda^2 - x_p - x_Q\\
	&y_{P+Q}=-(\lambda(\lambda^2- x_P - x_Q) +y_P-\lambda x_P)
\end{aligned}
\right.
$
$\left\{
\begin{aligned}
	&x=\lambda^2 - x_p - x_Q\\
	&y=-\lambda(\lambda^2-2x_P - x_Q) -y_P
\end{aligned}
\right.
$

$\text{Soit }\left\{
\begin{aligned}
	&x=\lambda^2 - x_p - x_Q\\
	&y=\lambda(x_P + x_Q - x_{P+Q}) -y_P
\end{aligned}
\right.
$


 \end{document}
 
 
 
 
 
 
 