\documentclass{beamer}
\usepackage{verbatim}
\usepackage{graphicx}
\usepackage{pgf}
\usepackage{tikz}
\usepackage{mathrsfs, alltt, amssymb, amsmath, amsfonts, stmaryrd, amsthm}
\usepackage{etex}
\usetikzlibrary{automata}
%\usetheme{Hannover}
\usetheme{Boadilla}
\setbeamertemplate{navigation symbols}{}
\usecolortheme{beaver}
\usepackage[applemac]{inputenc}
\usepackage[french]{babel}
\usepackage{pgf}
\usepackage{tikz}
\usepackage{bussproofs}
\usetikzlibrary{arrows,shapes}
\bibliographystyle{plain}


\newcommand{\restate}[1]{
\begin{center}\begin{minipage}[t]{0.80\linewidth}#1\end{minipage}\end{center}}

\title[Paulard versus ECM]{Projet de calcul formel : Algorithme ECM de factorisation d'entiers}
\author{Christophe Vuillot et Yannick Zakowski}
\date{}

\AtBeginSection[]{
   \begin{frame}
   \begin{center}{\Large Plan }\end{center}
   \tableofcontents[currentsection,hideothersubsections]
   \end{frame} 
}

\begin{document}

\begin{frame}
	\titlepage
	\begin{figure}
	\end{figure}
\end{frame}

\begin{frame}{L'algorithme $p-1$ de Pollard}

$N$ un entier � d�composer
\begin{itemize}
	\item Ne serait-il pas premier ? Non : on voudrait donc un bon candidat diviseur, par exemple un �l�ment $g$ non inversible sur $\mathbb{Z}/{N\mathbb{Z}}$
	\item Pour cela, tirons al�atoirement un $a$ major� par une borne de lissage bien choisie.
	\item Cherchons une puissance de $a$ dont le $pgcd$ avec $N$ est diff�rent de $1$. 
	\item C'est gagn� !
\end{itemize}

\end{frame}

\begin{frame}{Un algorithme efficace...}
	L'efficacit� de l'algorithme $p-1$ de Pollard repose sur la relativement forte densit� d'�l�ments non inversibles sur $\mathbb{Z}/{N\mathbb{Z}}.$
	\pause
	
	On peut d�montrer qu'il poss�de une complexit� en $O(n^{1/4}).$
\end{frame}

\begin{frame}{L'algorithme ECM}

	\begin{itemize}
	\item	Conserver cette id�e de candidat diviseur en tant qu'�l�ment non-inversible sur $\mathbb{Z}/{N\mathbb{Z}}$ 
	\item mais augmenter nos chances de le trouver.
\pause

\item L'id�e va �tre de calculer � pr�sent sur une courbe elliptique !
\end{itemize}
\end{frame}

\begin{frame}{La formule de la somme de deux points}
	
 Consid�rons la courbe elliptique $\mathcal{E}$ d'�quation $y^2=x^3+ax+1.$
 
 Soit $P=(x_P,\; y_P)$ et $Q =(x_Q, \; y_Q)$ deux points de $\mathcal{E}.$\\
 
 \vspace{1cm}
 Montrons que
  $\left\{
\begin{aligned}
	&x_{P+Q}=\lambda^2 - x_p - x_Q\\
	&y_{P+Q}=\lambda(x_P + x_Q - x_{P+Q}) -y_P
\end{aligned}
\; avec \lambda=\frac{y_P-y_Q}{x_P-x_Q}
\right.
$
 \vspace{1cm}
 
 La droite $(PQ)$ a pour �quation : 
 \[y=\lambda x+ y_P - \lambda x_P \text{ avec }\lambda = \frac{y_Q-y_P}{x_Q-x_P} \]
 \end{frame}
 
 \begin{frame}{La formule de la somme de deux points}
 Cherchons les coordonn�es du troisi�me point d'intersection de cette droite avec $\mathcal{E} : $\\
\begin{equation*}
\left\{
\begin{aligned}
	&y=\lambda x+ y_P - \lambda x_P \\
	&y^2=x^3+ax+1
\end{aligned}
\right.
\end{equation*}

Son abscisse v�rifie donc :
\begin{equation*}
\begin{aligned}
	&(\lambda x+ y_P - \lambda x_P)^2=x^3+ax+1\\
	&x^3 - \lambda^2x^2+(a+2\lambda^2x_P-2\lambda y_P)x+1=0\\
\end{aligned}
\end{equation*}
\begin{equation*}
\begin{aligned}
	&x_P\text{ �tant solution }\\&(x-x_P)(x^2+(x_P-\lambda^2)x+(a+2\lambda^2x_P-2\lambda y_P+x_P(x_P-\lambda^2))=0\\
	&x_Q\text{ �tant solution }\\&(x-x_P)(x-x_Q)(x+x_P+x_Q-\lambda^2)=0
\end{aligned}
\end{equation*}
\end{frame}

\begin{frame}{La formule de la somme de deux points}
\centering
$
\text{D'o� }\left\{
\begin{aligned}
	&x_{P+Q}=\lambda^2 - x_p - x_Q\\
	&y_{P+Q}=-(\lambda(\lambda^2- x_P - x_Q) +y_P-\lambda x_P)
\end{aligned}
\right.
$
\vspace{1cm}

$\left\{
\begin{aligned}
	&x=\lambda^2 - x_p - x_Q\\
	&y=-\lambda(\lambda^2-2x_P - x_Q) -y_P
\end{aligned}
\right.
$
\vspace{1cm}

$\text{Soit }\left\{
\begin{aligned}
	&x=\lambda^2 - x_p - x_Q\\
	&y=\lambda(x_P + x_Q - x_{P+Q}) -y_P
\end{aligned}
\right.
$

\end{frame}

\begin{frame}{L'id�e}
	\begin{itemize}
		\item Soit $P$ un point sur notre courbe.
		\item Calculons ses multiples sur $E(\mathbb F_N)$.
		\item Ce n'est pas une v�ritable courbe elliptique, on est amen� � trouver un �l�ment non-inversible de $\mathbb{Z}/N\mathbb{Z}$
		\item Nous retombons dans le cadre de Pollard
	\end{itemize}
	
\end{frame}

\begin{frame}{Complexit�}
	On peut montrer que la complexit� ne d�pend que de $p$ et non de $n$, selon :
	\centering $O(e^{(\sqrt(2)+o(1))\sqrt{\log{p}\log{\log{p}}}})$
\end{frame}



\end{document}